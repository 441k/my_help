\documentclass[12pt,a4paper]{jsarticle}
\usepackage[dvipdfmx]{graphicx}
\usepackage[dvipdfmx]{color}
\usepackage{listings}
% to use japanese correctly, install jlistings.
\lstset{
  basicstyle={\small\ttfamily},
  identifierstyle={\small},
  commentstyle={\small\itshape\color{red}},
  keywordstyle={\small\bfseries\color{cyan}},
  ndkeywordstyle={\small},
  stringstyle={\small\color{blue}},
  frame={tb},
  breaklines=true,
  numbers=left,
  numberstyle={\scriptsize},
  stepnumber=1,
  numbersep=1zw,
  xrightmargin=0zw,
  xleftmargin=3zw,
  lineskip=-0.5ex
}
\lstdefinestyle{customCsh}{
  language={csh},
  numbers=none,
}
\lstdefinestyle{customRuby}{
  language={ruby},
  numbers=left,
}
\lstdefinestyle{customTex}{
  language={tex},
  numbers=none,
}
\lstdefinestyle{customJava}{
  language={java},
  numbers=left,
}
\begin{document}
\title{卒業論文\\
\vspace{4cm} ユーザメモソフトmy\_helpの開発}
\author{ 関西学院大学 理工学部 情報科学科\\\\2535 那須比呂貴}
\date{\vspace{3cm} 2017年  3月\\
\vspace{3cm} 指導教員  西谷 滋人 教授}
\maketitle
\tableofcontents

\title{ユーザメモソフトmy\_helpの開発}
\author{2535 那須比呂貴}
\date{}
\maketitle
\tableofcontents
 本研究ではユーザメモソフトであるmy\_helpの開発において,BDDを取り入れることによりmy\_helpの向上を目指した.

 my\_helpとは,ユーザメモソフトであり,user独自のhelpを作成・提供することができるgemである.
しかし,これらの仕様方法を初心者が理解すること自体に時間がかかってしまうという問題点がある.
そこで,cucumberを用いる.cucumberはRubyでBDDを実践するために用意された環境である.
したがって,cucumberは振る舞いをチェックするために記述するが,そこで日本語がそのまま用いることが可能であるため,その記述を読むだけで,my\_helpの振る舞いを理解することが可能となる.

 cucumberは実際にソフトウェア開発の現場において,ユーザーとプログラマがお互いの意思疎通のために利用される.
テストはプログラムがチェックしてくれるが,記述は人間が理解できなければならない.
この二つの要求を同時に叶えようというのが,BDDの基本思想である.
これらは,研究室の知識を定着させることに有益であり,研究室の役に立つと考えた.


\section{目的}
プログラム開発では,統合開発環境がいくつも用意されているが,
多くの現場では,terminal上での開発が一般的である.
ところが,プログラミング初心者はterminal上でのcharacter user interface(CUI)を苦手としている.
プログラミングのレベルが上がるに従って,
shell commandやfile directory操作, process制御にCUIを使うことが常識となる.
この不可欠なCUIスキルの習得を助けるソフトとして,ユーザメモソフトmy\_helpがruby gemsに置かれている.
このcommand line interface(CLI)で動作するソフトは,helpをterminal上で簡単に提示するものである.
また,初心者が自ら編集することによって,すぐに参照できるメモとしての機能を提供している.
これによって,terminal上でちょっとした調べ物ができるため,作業や思考が中断することなく
プログラム開発に集中できることが期待でき,初心者のスキル習得が加速することが期待できる.

しかし,このRubyで書かれたソフトは動作するが,テストが用意されていない.
今後ソフトを進化させるために共同開発を進めていくには,
仕様や動作の標準となるテスト記述が不可欠となる.

本研究の目的は,ユーザメモソフトであるmy\_helpのテスト開発である.
ここでは,テスト駆動開発の中でも,ソフトの振る舞いを記述する
Behavior Driven Development(BDD)に基づいてテストを記述していく.
そこで,my\_helpがどのような振る舞いをするのかをCucumberとRSpecを用いてBDDでコードを書いていく.
Cucumberは自然言語で振る舞いを記述することができるため,ユーザにとって,わかりやすく振る舞いを確認することができる.


\section{RSpec}
 RSpecはSteven Bakerによって2005年に作成されました.StevenはAslak HellesoyからBDDのことを聞いていました.BDDという考え方が知られるようになった頃,AslakはDan Northとともにあるプロジェクトに取り組んでいました.SmalltalkやRubyといった言語を使って,振る舞いに注目することを促す新しいTDDフレームワークをもっと自由に探求してもよいはずだとDave Astelsが提案したとき,Stevenはすでにその考えに共感を抱いてました.そしてRSpecが誕生したのです.
 構文の細かい部分はStevenが作成したRSpecの最初のバージョンから進化していますが,基本的な前提は同じです.私たちはRSpecを使って実行可能なサンプルから記述します.これらのサンプルは,制御されたコンテキストにおいて期待される振る舞いを表すほんのわずかなコードで構成されます.
それは次のようになります.
\begin{quote}\begin{verbatim}
describe MovieList do
  context "when first created" do
    it "is empty" do
      movie-list - MovieList.new
      movie_list.should be_empty
    end
  end
end

\end{verbatim}\end{quote}
 it()メソッドは,MovieListが作成されたコンテキストにおいて,MovieListの振る舞いのサンプルを作成します.movie\_list.should be\_emptyという式については説明するまでもないでしょう.声に出して読んでみればわかります.be\_emptyがmovie\_listとどのようにやり取りするかについては,後に説明します.
 シェルとrspecコマンドを使ってこのコードを実行すると,次のような出力が得られます.
\begin{quote}\begin{verbatim}
MovieList when first created
   is empty
\end{verbatim}\end{quote}
 コンテキストとサンプルをさらに追加すると,結果として得られる出力がMovieListオブジェクトの仕様にだんだん近づいていきます.
\begin{quote}\begin{verbatim}
MovieList hen first created
   is empty

MovieList with 1 item
   is not empty
   includes that item

\end{verbatim}\end{quote}
 もちろん,ここで述べているのはシステムではなくオブジェクトの仕様です.RSpecを使ってアプリケーションの振る舞いを指定することは可能であり,多くの開発者がそうしています.しかし.アプリケーションの振る舞いを指定するには,何かもっと大きな流れで意思を伝えるものが必要です.
 そこで.Cucumberを使うことにします[1, pp6-7].


\section{BDD}
ビヘイビア駆動開発(Behaviour-Driven Development : BDD)は,テスト駆動開発(Test-Driven Development : TDD)の工程への理解を深め,それをうまく説明しようとして始まりました.

BDDは構造ではなく振る舞いに焦点を合わせます.
それは開発のすべてのレベルでいっかんしてそうなります.
2つの都市の間の距離を計算するオブジェクトのことであっても,サードパーティのサービスに検索を委任する別のオブジェクトのことであっても,
あるいはユーザーが無効なデータを入力したときにフィードバックを提供する別の画面であっても,それはすべて振る舞いなのです.
これを飲み込んでしまえば,コードに取り組むときの考え方が変わります.
オブジェクトの構造よりも,ユーザーとシステムの間でのやり取り,つまりオブジェクトの間でのやり取りについて考えるようになります.

ソフトウェア開発チームが直面する問題のほとんどは,コミュニケーションの問題であると考えています.
BDDの目的は,ソフトウェアが使われる状況を説明するための言語を単純かすることで,コミュニケーションを後押しすることです.
つまり,あるコンテキストで(Given),あるイベントが発生すると(When),ある結果が期待されます(Then).
BDDにおけるGiven, When,Thenの3つの単語は,アプリケーションやオブジェクトを,それらの振る舞いに関係なく表現するために使われる単純な単語です.
ビジネスアナリスト,テスト担当者,開発者は皆,それらをすぐに理解します.
これらの単語はCucumberの言語に直接埋め込まれています[1].

BDDサイクルの図を以下に示します[1].
\verb|{{attach_view(my_help_nasu.001.jpg)}}|
\verb|{{attach_view(my_help_nasu1.001.jpg)}}|


\section{Cucumberとfeaturesの説明}
BDDはフルスタックのアジャイル開発技法です.BDDはATDP(Acceptance Test-Driven Planning)と呼ばれるAcceptance TDDの一種を含め,エクストリームプログラミングからヒントを得ています.
ATDPでは,顧客受け入れテストを導入し,それを主体にコードの開発を進めて行きます.
それらは顧客と開発チームによる共同作業の結果であることが理想的です.
開発チームによってテストが書かれた後,顧客がレビューと承認を行うこともあります.
いずれにしても,それらのテストは顧客と向き合うものなので,顧客が理解できる言語とフォーマットで表現されていなければなりません.
Cucumberを利用すれば,そのための言語とフォーマットを手に入れることができます.
Cucumberは,アプリケーションの機能とサンプルシナリオを説明するテキストを読み取り,そのシナリオの手順に従って開発中のコードとのやり取りを自動化します[1].
下記に例を示しています.
\begin{lstlisting}[style=customRuby]
# language: ja

機能: todoの更新を行う
todoは更新していくものであり,新しく書いたり終わったものを消したいのでバッ\
クアップをとって,過去のtodoを残しておく

シナリオ: コマンドを入力してtodoを更新していく
          前提 todoを編集したい
          もし "my_todo --edit"と入力する
          ならば editが開かれる
          かつ 自分のtodoを書き込む

シナリオ: コマンドを入力してバックアップをとる
          前提 todoの編集が終わった
          もし "my_todo --store [item]"と入力する
          ならば itemのバックアップを取る
\end{lstlisting}
このように日本語でシナリオを書くことができ,顧客にもわかりやすく,開発者も書きやすくなっている.

featureファイルに書くキーワードは下記の通りになっている.

\begin{table}[htbp]\begin{center}
\caption{}
\begin{tabular}{lll}
\hline
feature  &"フィーチャ", "機能"     \\ \hline
background  &"背景"              \\
scenario  &"シナリオ"            \\
scenario\_outline   &"シナリオアウトライン", "シナリオテンプレート", "テンプレ", "シナリオテンプレ"   \\
examples  &"例", "サンプル"       \\
given   &"* ", "前提"        \\
when   &"* ", "もし"        \\
then   &"* ", "ならば"       \\
and    &"* ", "かつ"        \\
but    &"* ", "しかし", "但し", "ただし"  \\
given (code)   &"前提"              \\
when (code)   &"もし"              \\
then (code)   &"ならば"             \\
and (code)    &"かつ"              \\
but (code)    &"しかし", "但し", "ただし"   \\
\hline
\end{tabular}
\label{default}
\end{center}\end{table}
%for inserting separate lines, use \hline, \cline{2-3} etc.


\section{my\_helpについて}
my\_helpは本研究室の西谷が開発したものです.

以下はmy\_helpのREADMEです[2].

CUI(CLI)ヘルプのUsage出力を真似て,user独自のhelpを作成・提供するgem.
\begin{enumerate}
\item 問題点
\end{enumerate}
CUIやshell, 何かのプログラミング言語などを習得しようとする初心者は,
commandや文法を覚えるのに苦労します.少しのkey(とっかかり)があると
思い出すんですが,うろ覚えでは間違えて路頭に迷います.問題点は,
- manは基本的に英語
- manualでは重たい
- いつもおなじことをwebで検索して
- 同じとこ見ている
- memoしても,どこへ置いたか忘れる

などです.
\begin{enumerate}
\item 特徴
\end{enumerate}
これらをgem環境として提供しようというのが,このgemの目的です.
仕様としては,
- userが自分にあったmanを作成
- 雛形を提供
\begin{quote}\begin{verbatim}
 - おなじformat, looks, 操作, 階層構造
\end{verbatim}\end{quote}
- すぐに手が届く
- それらを追加・修正・削除できる

hikiでやろうとしていることの半分くらいはこのあたりのことなの
かもしれません.memoソフトでは,検索が必要となりますが,my\_helpは
key(記憶のとっかかり)を提供することが目的です.

\section{my\_helpのインストール}
\subsection{githubに行ってdaddygongonのmy\_helpをforkする}\begin{enumerate}
\item git clone git@github.com:daddygongon/my\_help.git
\item cd my\_help
\item rake to\_yml
\item rake clean\_exe
\item [sudo] bundle exec exe/my\_help -m
\item source ~/.zshrc or source ~/.cshrc
\item my\_help -l
\item rake add\_yml
\end{enumerate}
\section{my\_helpの更新}
\subsection{gitを用いてmy\_helpを新しくする.}\begin{enumerate}
\item git remote -vをする(remoteの確認).
\item (upstreamがなければ)git remote add upstream git@github.com:gitname/my\_help.git
\item git add -A
\item git commit -m 'hogehoge'
\item git push upstream master(ここで自分のmy\_helpをupstreamに送っとく)
\item git pull origin master(新しいmy\_helpを取ってくる)
\end{enumerate}
\subsection{次にとってきた.ymlを~/.my\_helpにcpする.}\begin{enumerate}
\item cd my\_helpでmy\_helpに移動.
\item cp hogehoge.yml ~/.my\_help
\end{enumerate}
それを動かすために
(sudo)bundle exec ruby exe/my\_help -mをする.

ここで過去にsudoをした人はpermissionがrootになっているので,sudoをつけないとerrorが出る.

(sudoで実行していたら権限がrootに移行される)

新しいターミナルを開いて動くかチェックする.


下記にfeaturesの記述を示す.
\verb|my_help_nasu_features|


\section{結果}
CucumberとRSpecを用いてBDDでmy\_helpのテスト開発を進めて行きました.
ここでは,焦点を合わせたmy\_helpの中での一つの振る舞いである
「todoの更新」を例として詳しく見て行きます.

\subsection{todoの更新マニュアル}
最初に,todoを更新するときの手順を示します.

\begin{enumerate}
\item my\_todo --editを入力して~/.my\_help/my\_todo.ymlを開く
\item editorでtodoを書き込む(今週やることならweeklyというitemを作ってそこに書き込む)
\item 保存して~/.my\_help/my\_todo.ymlを閉じる
\item my\_todoと打ち込んで更新されていたら完成
\item my\_todo --store [item]を入力してitemのバックアップをとる
\end{enumerate}
この振る舞いがきちんとできているのかを実際にテスト構築して行きます.

\subsection{Cucumber}
以下はtodoの更新を行うときのfeatureです.
まず,適当なディレクトリにfeaturesというディレクトリを作成します.
次に先ほど作成した,featuresディレクトリにmy\_todo.featureを作成します.
\begin{lstlisting}[basicstyle=\scriptsize]
# language: ja 

機能: todoの更新を行う
todoは更新していくものであり,新しく書いたり終わったものを
消したいのでバックアップをとって,過去のtodoを残しておく


 シナリオ: コマンドを入力してtodoを更新していく
          前提 todoを編集したい
          もし "my_todo --edit"と入力する
          ならば editが開かれる
          かつ 自分のtodoを書き込む

シナリオ: コマンドを入力してバックアップをとる
          前提 todoの編集が終わった
          もし "my_todo --store [item]"と入力する
          ならば itemのバックアップを取る

\end{lstlisting}
機能とは,このシステムの機能のことを記述します.ここでは,todoを更新するシステムですので,「todoの更新を行う」です.
機能の下には,機能の補足説明を記述します.機能の補足説明では,ルールがないので自分がわかりやすいように,記述するのが常識です.
シナリオは,その名の通りtodoを更新する時のユーザの行動やシステム振る舞いを前提,もし,ならば,かつ,しかしに分類して記述します.
シナリオは,一つの機能に対して複数書くことが可能です.ここでは、「todoを更新する」と「バックアップをとる」という二つの
シナリオをたてています.

ここまでfeatureが記述できたら,次はcucumberコマンドを実行してみます.
コマンドは以下の通りです.
 /Users/nasubi/nasu% cucumber features/my\_todo.feature 
featuresディレクトリにあるmy\_todo.featureファイルをcucumberで実行するという意味です.

実行すると以下のようになります.
\begin{lstlisting}[basicstyle=\scriptsize]
# language: ja
機能: todoの更新を行う
todoは更新していくものであり,新しく書いたり終わったものを消したいのでバックアップをとって,過去のtodoを残しておく

  シナリオ: コマンドを入力してtodoを更新していく # features/my_todo.feature:6
    前提todoを編集したい             # features/my_todo.feature:7
    もし"my_todo --edit"と入力する  # features/my_todo.feature:8
    ならばeditが開かれる             # features/my_todo.feature:9
    かつ自分のtodoを書き込む           # features/my_todo.feature:10

  シナリオ: コマンドを入力してバックアップをとる          # features/my_todo.feature:12
    前提todoの編集が終わった                  # features/my_todo.feature:13
    もし"my_todo --store [item]"と入力する # features/my_todo.feature:14
    ならばitemのバックアップを取る               # features/my_todo.feature:15

2 scenarios (2 undefined)
7 steps (7 undefined)
0m0.080s

You can implement step definitions for undefined steps with these snippets:

前提(/^todoを編集したい$/) do
  pending # Write code here that turns the phrase above into concrete actions
end

もし(/^"([^"]*)"と入力する$/) do |arg1|
  pending # Write code here that turns the phrase above into concrete actions
end

ならば(/^editが開かれる$/) do
  pending # Write code here that turns the phrase above into concrete actions
end

ならば(/^自分のtodoを書き込む$/) do
  pending # Write code here that turns the phrase above into concrete actions
end

前提(/^todoの編集が終わった$/) do
  pending # Write code here that turns the phrase above into concrete actions
end

ならば(/^itemのバックアップを取る$/) do
  pending # Write code here that turns the phrase above into concrete actions
end
\end{lstlisting}
ここでは,2つのscenarioと7つのstepが失敗しています.
まだstep定義を記述していないので当たり前です.

一度cucumberを実行したのには理由があります.
featureを書いた時点でcucumberを実行すると,ステップ定義の元となるコマンドを,cucumberが自動的に作成してくれるからです.

以下がcucumberから出力されたステップ定義の元となる部分です.
\begin{quote}\begin{verbatim}
You can implement step definitions for undefined steps with these snippets:

前提(/^todoを編集したい$/) do
  pending # Write code here that turns the phrase above into concrete actions
end

もし(/^"([^"]*)"と入力する$/) do |arg1|
  pending # Write code here that turns the phrase above into concrete actions
end

ならば(/^editが開かれる$/) do
  pending # Write code here that turns the phrase above into concrete actions
end

ならば(/^自分のtodoを書き込む$/) do
  pending # Write code here that turns the phrase above into concrete actions
end

前提(/^todoの編集が終わった$/) do
  pending # Write code here that turns the phrase above into concrete actions
end

ならば(/^itemのバックアップを取る$/) do
  pending # Write code here that turns the phrase above into concrete actions
end
\end{verbatim}\end{quote}
これをコピーして,featuresディレクトリの中でstep\_definitionsディレクトリを作成し,その中にmy\_todo\_spec.rbを作成し,そこに貼付けます.

ここでもう一度cucumberを実行してみると
\begin{quote}\begin{verbatim}
/Users/nasubi/nasu% cucumber features/my_todo.feature 
# language: ja
機能: todoの更新を行う
todoは更新していくものであり,新しく書いたり終わったものを消したいのでバックアップをとって,過去のtodoを残しておく

  シナリオ: コマンドを入力してtodoを更新していく # features/my_todo.feature:6
    前提todoを編集したい             # features/step_definitions/my_todo_spec.rb:1
      TODO (Cucumber::Pending)
      ./features/step_definitions/my_todo_spec.rb:2:in `/^todoを編集したい$/'
      features/my_todo.feature:7:in `前提todoを編集したい'
    もし"my_todo --edit"と入力する  # features/step_definitions/my_todo_spec.rb:5
    ならばeditが開かれる             # features/step_definitions/my_todo_spec.rb:9
    かつ自分のtodoを書き込む           # features/step_definitions/my_todo_spec.rb:13

  シナリオ: コマンドを入力してバックアップをとる          # features/my_todo.feature:12
    前提todoの編集が終わった                  # features/step_definitions/my_todo_spec.rb:17
      TODO (Cucumber::Pending)
      ./features/step_definitions/my_todo_spec.rb:18:in `/^todoの編集が終わった$/'
      features/my_todo.feature:13:in `前提todoの編集が終わった'
    もし"my_todo --store [item]"と入力する # features/step_definitions/my_todo_spec.rb:5
    ならばitemのバックアップを取る               # features/step_definitions/my_todo_spec.rb:21

2 scenarios (2 pending)
7 steps (5 skipped, 2 pending)
0m0.045s

\end{verbatim}\end{quote}
と変化が出てきます.
\begin{quote}\begin{verbatim}
2 scenarios (2 pending)
7 steps (5 skipped, 2 pending)
\end{verbatim}\end{quote}
これは2つのシナリオの内2つがpendingであり,7つのstepの内2つがpendingで5つがskipしたことを表しています.
step\_definitionsのmy\_todo\_spec.rbのpending部分を書き換えて,step\_definitionsの記述を進めて行きます.

cucumberが成功すると下記のような結果となります.
\begin{lstlisting}[style=]
/Users/nasubi/my_help% cucumber features/my_todo.feature 
# language: ja
機能: todoの更新を行う
todoは更新していくものであり,新しく書いたり終わったものを消したいのでバックアップをとって,過去のtodoを残しておく

  シナリオ: コマンドを入力してtodoを更新していく # features/my_todo.feature:6
    前提todoを編集したい             # features/step_definitions/my_todo_spec.rb:2
    もし"my_todo --edit"と入力する  # features/step_definitions/my_todo_spec.rb:6
    ならばeditが開かれる             # features/step_definitions/my_todo_spec.rb:10
    かつ自分のtodoを書き込む           # features/step_definitions/my_todo_spec.rb:14

  シナリオ: コマンドを入力してバックアップをとる          # features/my_todo.feature:12
    前提todoの編集が終わった                  # features/step_definitions/my_todo_spec.rb:18
    もし"my_todo --store [item]"と入力する # features/step_definitions/my_todo_spec.rb:6
    ならばitemのバックアップを取る               # features/step_definitions/my_todo_spec.rb:22

2 scenarios (2 passed)
7 steps (7 passed)
0m0.030s

\end{lstlisting}
\subsection{RSpec}
次にRSpecを使って実際にtodoを更新する振る舞いをするコード書いていく.

そのための準備として,まずspecというディレクトリを作成し,my\_todoというサブディレクトリを追加する.
次に,このサブディレクトリにtodo\_spec.rbというファイルを追加する.
作業を進める過程で,lib/my\_todo/my\_todo.rbソースファイルとspec/my\_todo/todo\_spec.rbスペックファイルが1対1に対応するといった要領で,
並列のディレクトリ構造を築いていく.
この機能はmy\_help --editと入力されれば,~/.my\_help/my\_todo.ymlが開かれるのでその振る舞いをするコードを書きます.
まずtodo\_spec.rbは下記の通りになります
\begin{lstlisting}[style=]
require 'spec_helper'


module Mytodo
  describe Todo do
    describe "#open" do
      it "open file my_todo.yml" 
    end
  end
end

\end{lstlisting}
describe()メソッドは,RSpecのAPIにアクセスしてRSpec::Core::ExampleGroupのサブクラスを返します.
ExampleGroupクラスはオブジェクトに期待される振る舞いのサンプルを示すグループです.
it()メソッドはサンプルを作成します.

このスペックを実行するために,specディレクトリにspec\_helper.rbを追加します.
中身は下記の通りです.
\begin{lstlisting}[style=]
$LOAD_PATH.unshift File.expand_path('../../lib', __FILE__)
require 'my_help'
require 'todo'
\end{lstlisting}
これで事前準備は完成でコードを書いていきます.

完成したコードを下記の通りです.
\begin{lstlisting}[style=]
require 'spec_helper'


module Mytodo
  describe Todo do
    describe "#open" do
      it "open file my_todo.yml" do
          system("emacs ~/.my_help/my_todo.yml")
      end
    end
  end
end

\end{lstlisting}
これでrspecを実行すると以下のような結果が表示される.
\begin{lstlisting}[style=]
/Users/nasubi/my_help% rspec spec --color

my_help command
  version option
    should be successfully executed
    should have output: "my_help 0.4.3"
  help option
    should be successfully executed
    should have output: "Usage: my_help [options]\n    -v, --version                    show program Version.\n    -l, --list...              install local after edit helps\n        --delete NAME                delete NAME help"

Mytodo::Todo
  #open
    open file my_todo.yml

Finished in 3.87 seconds (files took 0.30703 seconds to load)
5 examples, 0 failures

\end{lstlisting}
これでtodoを更新するときの振る舞いのテストはうまく成功した.

この方法でBDDでmy\_helpのテスト開発を行い,仕様を決定していった.


\begin{enumerate}
\item The RSpec Book 著者:David Chelimsky Dave Astels Zach Dennis ほか 翻訳:株式会社クイーブ 監修:株式会社クイーブ 角谷信太郎 豊田裕司.
\item Shigeot R. Nishitani, my\_helpのREADME, \verb|http://www.rubydoc.info/gems/my_help/0.4.3|.
\end{enumerate}
\end{document}
