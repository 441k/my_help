\section{my\_helpについて}
my\_helpは本研究室の西谷が開発したものです.

以下はmy\_helpのREADMEです[2].

CUI(CLI)ヘルプのUsage出力を真似て,user独自のhelpを作成・提供するgem.
\begin{enumerate}
\item 問題点
\end{enumerate}
CUIやshell, 何かのプログラミング言語などを習得しようとする初心者は,
commandや文法を覚えるのに苦労します.少しのkey(とっかかり)があると
思い出すんですが,うろ覚えでは間違えて路頭に迷います.問題点は,
- manは基本的に英語
- manualでは重たい
- いつもおなじことをwebで検索して
- 同じとこ見ている
- memoしても,どこへ置いたか忘れる

などです.
\begin{enumerate}
\item 特徴
\end{enumerate}
これらをgem環境として提供しようというのが,このgemの目的です.
仕様としては,
- userが自分にあったmanを作成
- 雛形を提供
\begin{quote}\begin{verbatim}
 - おなじformat, looks, 操作, 階層構造
\end{verbatim}\end{quote}
- すぐに手が届く
- それらを追加・修正・削除できる

hikiでやろうとしていることの半分くらいはこのあたりのことなの
かもしれません.memoソフトでは,検索が必要となりますが,my\_helpは
key(記憶のとっかかり)を提供することが目的です.

\section{my\_helpのインストール}
\subsection{githubに行ってdaddygongonのmy\_helpをforkする}\begin{enumerate}
\item git clone git@github.com:daddygongon/my\_help.git
\item cd my\_help
\item rake to\_yml
\item rake clean\_exe
\item [sudo] bundle exec exe/my\_help -m
\item source ~/.zshrc or source ~/.cshrc
\item my\_help -l
\item rake add\_yml
\end{enumerate}
\section{my\_helpの更新}
\subsection{gitを用いてmy\_helpを新しくする.}\begin{enumerate}
\item git remote -vをする(remoteの確認).
\item (upstreamがなければ)git remote add upstream git@github.com:gitname/my\_help.git
\item git add -A
\item git commit -m 'hogehoge'
\item git push upstream master(ここで自分のmy\_helpをupstreamに送っとく)
\item git pull origin master(新しいmy\_helpを取ってくる)
\end{enumerate}
\subsection{次にとってきた.ymlを~/.my\_helpにcpする.}\begin{enumerate}
\item cd my\_helpでmy\_helpに移動.
\item cp hogehoge.yml ~/.my\_help
\end{enumerate}
それを動かすために
(sudo)bundle exec ruby exe/my\_help -mをする.

ここで過去にsudoをした人はpermissionがrootになっているので,sudoをつけないとerrorが出る.

(sudoで実行していたら権限がrootに移行される)

新しいターミナルを開いて動くかチェックする.

