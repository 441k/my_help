\section{Cucumberとfeaturesの説明}
BDDはフルスタックのアジャイル開発技法です.BDDはATDP(Acceptance Test-Driven Planning)と呼ばれるAcceptance TDDの一種を含め,エクストリームプログラミングからヒントを得ています.
ATDPでは,顧客受け入れテストを導入し,それを主体にコードの開発を進めて行きます.
それらは顧客と開発チームによる共同作業の結果であることが理想的です.
開発チームによってテストが書かれた後,顧客がレビューと承認を行うこともあります.
いずれにしても,それらのテストは顧客と向き合うものなので,顧客が理解できる言語とフォーマットで表現されていなければなりません.
Cucumberを利用すれば,そのための言語とフォーマットを手に入れることができます.
Cucumberは,アプリケーションの機能とサンプルシナリオを説明するテキストを読み取り,そのシナリオの手順に従って開発中のコードとのやり取りを自動化します[1].
下記に例を示しています.
\begin{lstlisting}[style=customRuby]
# language: ja

機能: todoの更新を行う
todoは更新していくものであり,新しく書いたり終わったものを消したいのでバッ\
クアップをとって,過去のtodoを残しておく

シナリオ: コマンドを入力してtodoを更新していく
          前提 todoを編集したい
          もし "my_todo --edit"と入力する
          ならば editが開かれる
          かつ 自分のtodoを書き込む

シナリオ: コマンドを入力してバックアップをとる
          前提 todoの編集が終わった
          もし "my_todo --store [item]"と入力する
          ならば itemのバックアップを取る
\end{lstlisting}
このように日本語でシナリオを書くことができ,顧客にもわかりやすく,開発者も書きやすくなっている.

featureファイルに書くキーワードは下記の通りになっている.

\begin{table}[htbp]\begin{center}
\caption{}
\begin{tabular}{lll}
\hline
feature  &"フィーチャ", "機能"     \\ \hline
background  &"背景"              \\
scenario  &"シナリオ"            \\
scenario\_outline   &"シナリオアウトライン", "シナリオテンプレート", "テンプレ", "シナリオテンプレ"   \\
examples  &"例", "サンプル"       \\
given   &"* ", "前提"        \\
when   &"* ", "もし"        \\
then   &"* ", "ならば"       \\
and    &"* ", "かつ"        \\
but    &"* ", "しかし", "但し", "ただし"  \\
given (code)   &"前提"              \\
when (code)   &"もし"              \\
then (code)   &"ならば"             \\
and (code)    &"かつ"              \\
but (code)    &"しかし", "但し", "ただし"   \\
\hline
\end{tabular}
\label{default}
\end{center}\end{table}
%for inserting separate lines, use \hline, \cline{2-3} etc.

