\section{Cucumberとfeaturesの説明}
Cucumberが提供するBDDの内容をまとめると

\begin{quotation}
BDDはフルスタックのアジャイル開発技法です.BDDはATDP(Acceptance Test-Driven Planning)と呼ばれるAcceptance TDDの一種を含め,エクストリームプログラミングからヒントを得ています.ATDPでは,顧客受け入れテストを導入し,それを主体にコードの開発を進めて行きます.それらは顧客と開発チームによる共同作業の結果であることが理想的です.開発チームによってテストが書かれた後,顧客がレビューと承認を行うこともあります.いずれにしても,それらのテストは顧客と向き合うものなので,顧客が理解できる言語とフォーマットで表現されていなければなりません.Cucumberを利用すれば,そのための言語とフォーマットを手に入れることができます.Cucumberは,アプリケーションの機能とサンプルシナリオを説明するテキストを読み取り,そのシナリオの手順に従って開発中のコードとのやり取りを自動化します[1, 7pp.].

\end{quotation}
と記されている.

下記にmy\_todoに対するfeaturesファイルの具体例を示す.
\begin{lstlisting}[style=customRuby]
# language: ja

機能: todoの更新を行う
todoは更新していくものであり,新しく書いたり終わったものを消したいのでバッ\
クアップをとって,過去のtodoを残しておく

シナリオ: コマンドを入力してtodoを更新していく
          前提 todoを編集したい
          もし "my_todo --edit"と入力する
          ならば editが開かれる
          かつ 自分のtodoを書き込む

シナリオ: コマンドを入力してバックアップをとる
          前提 todoの編集が終わった
          もし "my_todo --store [item]"と入力する
          ならば itemのバックアップを取る
\end{lstlisting}
このように日本語でシナリオを書くことができ,顧客にもわかりやすく,開発者も書きやすくなっている.

ファイルの先頭で,
\begin{quote}\begin{verbatim}
# language: ja
\end{verbatim}\end{quote}
と記すと日本語のkeywordが認識される.もし英語で書いたら,
\begin{quote}\begin{verbatim}
function:...

\end{verbatim}\end{quote}
となる.

featureファイルで使えるkeywordの対応は下記の通りになっている.

\begin{table}[htbp]\begin{center}
\caption{}
\begin{tabular}{lll}
\hline
feature  &"フィーチャ", "機能"     \\ \hline
background  &"背景"              \\
scenario  &"シナリオ"            \\
scenario\_outline   &"シナリオアウトライン", "シナリオテンプレート", "テンプレ", "シナリオテンプレ"   \\
examples  &"例", "サンプル"       \\
given   &"* ", "前提"        \\
when   &"* ", "もし"        \\
then   &"* ", "ならば"       \\
and    &"* ", "かつ"        \\
but    &"* ", "しかし", "但し", "ただし"  \\
given (code)   &"前提"              \\
when (code)   &"もし"              \\
then (code)   &"ならば"             \\
and (code)    &"かつ"              \\
but (code)    &"しかし", "但し", "ただし"   \\
\hline
\end{tabular}
\label{default}
\end{center}\end{table}
%for inserting separate lines, use \hline, \cline{2-3} etc.

\section{下記にCucumberとRSpecのインストール方法を示す[1, pp11-12].}
\subsection{インストール}
\subsubsection{まずrspecとcucumberをgemでinstallする}
\begin{enumerate}
\item gem install rspec --version 2.0.0
\item rspec --helpと入力して
\end{enumerate}\begin{quote}\begin{verbatim}
/Users/nasubi/nasu% rspec --help
Usage: rspec [options] [files or directories]
\end{verbatim}\end{quote}
のような表示がされていればinstallができている.

\begin{enumerate}
\item gem install cucumber --version 0.9.2
\item cucumber --helpと入力して
\end{enumerate}\begin{quote}\begin{verbatim}
cucumber --help
Usage: cucumber [options] [ [FILE|DIR|URL][:LINE[:LINE]*] ]+
\end{verbatim}\end{quote}
のような表示がされていればinstallできている.

