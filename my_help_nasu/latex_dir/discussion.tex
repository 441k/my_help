\section{結果と今後の課題}
\subsection{本研究の成果}
my\_helpのテスト開発が成功したため,仕様や動作の標準が確定しました.
私は,my\_helpは何度も使用することで,どのように進化させれば便利なのかがわかってくると考えているので,my\_helpはまだまだ進化する機会が出てくると思います.
つまり,テスト開発をしたことで,今後my\_helpを進化させるための共同開発が円滑に進める手助けになると推測します.
また,my\_helpの進化の開発だけに関わらず,my\_helpの使用方法が明確になったことで,私が作成したmy\_helpのfeaturesを読めば初心者でもmy\_helpの振る舞いが容易に理解できます.
my\_helpは本研究室で今後使われていくと予想し,本研究はこれからの研究の手助けになると考えています.
今後の本研究室の研究生が,CUIなどの習得に時間をかけずにプログラミングに集中でき,研究の質が向上すると予想できます.

\subsection{今後の課題}
my\_helpのテスト開発は成功したが,以下のような問題点が出てきました.

\begin{enumerate}
\item Cucumberを記述できるようになるまで時間がかかる.
\end{enumerate}
日本語で書くことができるので,記述されているコードを理解するのは容易にできました.しかし,日本語で書くことができるので容易に記述することも可能であると考えていましたが,The RSpec Bookに記載されているテンプレートが少なく,理解が難しく感じました.自力で作成することはできますが,やはり時間がかかってしまうと考えます.ネットで検索しても,The RSpec Bookと類似したシナリオが多いのも問題の一つであり,書き方のテンプレートがもっと必要です.地道にCucumberを使用して,データベースのようなものを作成するのがよいと考えました.また,BDDではCucumberでエラーが出ている状況が正しく,Cucumberを放置してRSpecへと進みます.今までのプログラミングとは少し違う感覚に陥り,不安がよぎります.これは慣れるまでの辛抱ではありますが,この問題も時間との戦いです.

\begin{enumerate}
\item BDDを行うときにrubyの知識が必要である.
\end{enumerate}
Cucumberのステップ定義やRSpecにおいてrubyを用いることが多々あります.スッテプ定義を記述するときに,あればよいと思うコードを書くのですが,具体的なプログラムを書くよりも難易度が高いと感じました.部分的なプログラムを作成するので,知識がないと正しいかどうかの判断が最後にしかできません.BDDを通してrubyを学べるという考え方もできますが,BDDでrubyを学ぶには物足りなさがあり,不向きです.したがって,BDDを使用する前にrubyの基礎の勉強は不可欠です.

