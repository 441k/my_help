
\section{新しいitemをspecific\_helpに追加}
シナリオ:コマンドを入力してspecific\_helpにitemを追加する

コマンド:emacs\_help --add [item]

specific\_helpとは,ユーザが作成するそれぞれのヘルプである.

ヘルプの内容は~/.my\_help/emacs\_help.ymlに元dataがある.

\section{全てのhelp画面の表示}
シナリオ:コマンドをニュ力してすべてのhelp画面を見る
コマンド:emacs\_help --all

\section{過去にバックアップしてあるitemのリストの表示}
シナリオ:コマンドを入力してバックアップのリストを見る
コマンド:emacs\_help --backup\_list

\section{helpコマンドの追加や削除,編集をするファイルの開示}
シナリオ:コマンドを入力してeditを開く
コマンド:emacs\_help --edit

元dataである~/.my\_help/emacs\_help.ymlを開く.

ここで編集を行い,emacsで開いているのでC-x,C-sで保存する.

\section{specific\_helpのitemの消去}
シナリオ:コマンドを入力してitemを消す

コマンド:emacs\_help --remove

\section{itemのバックアップ}
シナリオ:コマンドを入力してバックアップをとる

コマンド:emacs\_help --store [item]

\section{hikiへのformatの変更}
シナリオ:コマンドを入力してformatをhikiモードにする

コマンド:emacs\_help --to\_hiki

\section{todoの更新}
シナリオ1:コマンドを入力してtodoを更新する
シナリオ2:コマンドを入力してバックアップをとる

コマンド1:my\_todo --edit
コマンド2:my\_todo --store [item]

my\_todo --editで~/.my\_help/my\_todo.ymlを開く.

ここで編集を行い,emacsで開いているのでC-x,C-sで保存する.

my\_todo --store [item]でtodoのitemをバックアップとっておく.

この動作により過去のバックアップを閲覧することができ,どんどん更新することが可能である.

