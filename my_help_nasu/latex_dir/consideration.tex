\subsection{考察}
 今回の開発において,ユーザメモソフトであるmy\_helpのテスト開発は成功した.
つまり,my\_helpの仕様と動作の標準が確定したことで,今後のソフトを進化させるための共同開発がスムーズにいくと推測される.
先に述べた通り,ソフト開発は一人でせず,複数人で開発することが普通である.
その時に起こる障害として,意思の疎通ができていないことがあげられる.
振る舞いが標準化されていないと,どのような振る舞いをするのかが,プログラムを見るだけでは,ずれが生じてしまう.
また,開発者の意図が読めていないと,コードの意味も変わってくるので,テスト開発をして,仕様と動作の標準を確定することは重要であった.
加えて,初めてmy\_helpを使うユーザでも仕様と動作の標準がわかっていれば,短い時間でmy\_helpを使いこなせるし,そのことによって,ユーザそれぞれに自分にあったmy\_helpに変更することも容易になる.

