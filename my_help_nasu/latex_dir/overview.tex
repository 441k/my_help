 本研究ではユーザメモソフトであるmy\_helpの開発において,BDDを取り入れることによりmy\_helpの向上を目指した.

 my\_helpとは,ユーザメモソフトであり,user独自のhelpを作成・提供することができるgemである.
しかし,これらの仕様方法を初心者が理解すること自体に時間がかかってしまうという問題点がある.
そこで,cucumberを用いる.cucumberはRubyでBDDを実践するために用意された環境である.
したがって,cucumberは振る舞いをチェックするために記述するが,そこで日本語がそのまま用いることが可能であるため,その記述を読むだけで,my\_helpの振る舞いを理解することが可能となる.

 cucumberは実際にソフトウェア開発の現場において,ユーザーとプログラマがお互いの意思疎通のために利用される.
テストはプログラムがチェックしてくれるが,記述は人間が理解できなければならない.
この二つの要求を同時に叶えようというのが,BDDの基本思想である.
これらは,研究室の知識を定着させることに有益であり,研究室の役に立つと考えた.

